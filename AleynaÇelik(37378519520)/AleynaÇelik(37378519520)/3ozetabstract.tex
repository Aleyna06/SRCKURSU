\centerline{\bf �ZET}
\addcontentsline{toc}{section}{�ZET}
\begin{center}
\textbf{Projenin Amac�}
\end{center} 
 \mbox{Bilgisayar A�lar�n�n ilk dersinde  g�rd���m�z konular�n peki�tirilmesi.}\\
 

\begin{center}
\textbf{Projenin Kapsam�}
\end{center}
 \mbox{�lk derste ��renilen kavramlar�n tan�mlar�n� yapmak , D�z ve Cross kablo ba�lant�s�n�}
 \mbox{ger�eklemek.}

\begin{center}
\textbf{Sonu�lar}
\end{center}
 \mbox{D�z ve Cross kablo  ba�lant�lar RJ-45 konnekt�r� ile yap�ld� ve derste ��renilen terimler}
 \mbox{peki�tirildi.}
%%%%%%%%%%%%%%%%%%%%%%%%%%%%%%%%%%%%%%%ABSTRACT%%%%%%%%%%%%%%%%%%%%%%%%%%%%%%%%%%%%%%%%%%
\newpage
\centerline{\bf ABSTRACT}
\addcontentsline{toc}{section}{ABSTRACT}
\begin{center}
\textbf{Project Objective}
\end{center}

 \mbox{}\dotfill\\
  \mbox{}\dotfill
%Bilecik �eyh Edebali University Department of Computer Engineering students is studied. The aim of project is work to create a template for writing \LaTeX\ writing the final report template in the Project, to be written.

\begin{center}
\textbf{Scope of Project}
\end{center}
 \mbox{}\dotfill\\
  \mbox{}\dotfill
%Bilecik �eyh Edebali University Computer Engineering Department Bilecik need to create a project template Latex codes assignment, involves the use of. The first part of the project consists of two parts, \LaTeX's development have been studied and why it is preferred that the use of information provided in the information. Faculty of Engineering of the university for the second part, Sheikh Edebali Bilecik Project Paper \LaTeX\ codes and are included.

\begin{center}
\textbf{Results}
\end{center}
 \mbox{}\dotfill\\
  \mbox{}\dotfill
%As a result, Bilecik Sheikh Edebali University Computer Engineering
%Department prepared a document that students can build project reports into \LaTeX. 